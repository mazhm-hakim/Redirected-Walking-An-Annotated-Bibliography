\documentclass[12pt]{article}
\usepackage[a4paper, left=1in, right=1in, top=1in, bottom=1in]{geometry}
\usepackage[skip=15pt plus1pt, indent=60pt]{parskip}

\usepackage{soul}
\usepackage{enumitem}

\usepackage[style=apa]{biblatex}
\addbibresource{references.bib}

\usepackage{hyperref}
\hypersetup{
colorlinks,
citecolor=blue}

\newcommand\topPageBottom{
\noindent \hrulefill by {\scshape Md Azizul Hakim}\noindent \hrulefill \\
\normalsize
\texttt{mazhm.hakim@outlook.com} \\
\vspace{2.0em}
{\scshape January 2024}
}

\begin{document}

\begin{center}
\Huge
{\scshape Redirected Walking} \\
\LARGE
{\scshape An Annotated Bibliography} \\
\large
\topPageBottom
\end{center}

\vspace{2.0em}

\section{Annotated Bibliography}

\subsection{Redirected Walking in Place}

\textbf{\cite{razzaque2002redirected}}

It is the original paper on redirected walking. In this article, Razzaque et al. first introduced the redirected walking technique to allow participants to move about in a immersive virtual environment. The authors' aim was to develop the technique so that participants can turn within the VE by turning their body instead of using any hand control. This implementation of redirected walking relied on manipulation of the mapping between the user's real and virtual \emph{rotation}. This work demonstrated that small rotational changes are less likely to be detected and that the rotation is less likely to be noticed if the user's head is already turning. I cannot help reading this to create the foundational knowledge in this realm.

%Rotation, translation, curvature, bending

\subsection{Redirected Walking}
\textbf{\cite{razzaque2005redirected}}\textsuperscript{**}

This dissertation will be beneficial in forming the foundational knowledge on redirected walking. They includes \emph{virtual environments}, \emph{locomotion techniques}, \emph{statistical power analysis}, and \emph{Laplace analysis}. As a huge PhD report, I think I need more time to read this report and understand the mentioned topics in a whole.

\subsection{A taxonomy for deploying redirection techniques in immersive virtual environments}
\textbf{\cite{suma2012taxonomy}}

In this paper, Suma et al. presented a novel taxonomy that separates redirection techniques according to their geometric flexibility versus the likelihood that they will be noticed by the users. This taxonomy differentiate between redirection techniques that can be applied without the user's perceptibility (\emph{subtle}) and techniques that are perceptible by the user (\emph{overt}). Besides, based on their geometric applicability, it is also broadly divided into two categories: \emph{repositioning} techniques, \emph{reorienting} techniques. Repositioning works to manipulate correspondence between points in real and virtual worlds. It eventually compresses a larger virtual space into a smaller physical space. On the other hand, reorientation rotates the user's drift away from the boundaries of the physical space. This paper would be beneficial for me to understand the theoretical foundation for the development of automated redirection controllers that can dynamically apply a variety of techniques based upon the needs of the system and the current state of the user.

\subsection[Exploring large virtual environments with an HMD when physical space is limited]{Exploring large virtual environments\\ with an HMD when physical space is limited}
\textbf{\cite{williams2007exploring}}

...

\subsection{Analyses of human sensitivity to redirected walking}
\textbf{\cite{steinicke2008analyses}}

In this paper, the authors conducted a series of experiments in which they quantified how much humans can be redirected without observing inconsistencies between real and virtual motions. Their results summarize the corresponding gains in a practical useful range for their perceptibly. Reading this paper, I had got a handful idea about \emph{locomotion} and \emph{perception} in VE environments. I need to study this paper with more attention to understand the analysis of user's sensitivity to redirected walking manipulations.

\subsection{Leveraging change blindness for redirection in virtual environments}
\textbf{\cite{suma2011leveraging}}

In this paper, Suma et al. presented a novel redirection technique that exploits \emph{change blindness}, a perceptual phenomenon that occurs when a person fails to detect a visual change to an object or scene. Unlike previous RDW techniques, their approach does not introduce any visual-vestibular conflicts from manipulating the mapping between physical and virtual motions, nor does that require breaking presence to stop and explicitly reorient the user. Their aim was to change the location of doorways and corridors behind user's back and thereby manipulate their walking paths. This form of architectural manipulation enabled users to navigate a large dynamic virtual office building of approximately 219 sq. meters without leaving a tracking space of $4.3m \times 4.3m$.

\subsection{Impossible spaces: Maximizing natural walking in virtual environments with self-overlapping architecture}
\textbf{\cite{suma2012impossible}}

...

\subsection{Application of redirected walking in room-scale VR}
\textbf{\cite{langbehn2017application}}

In this paper, Langbehn et al. presented that they decreased the physical radius of at least 22 meters to apply undetectable redirected walking to room-scale VR. They achieved this by using bending gains and the detection thresholds for the human sensitivity to discrepancies between the curved paths in the real world and a curved path in the virtual environment. These ideas were previously published in their earlier paper (\cite{langbehn2017bending}). They aimed to enable redirected walking in room-scale VR without using interruptions or overt reorientation phases. It is another paper that I think will be crucial in forming my foundational knowledge.

\subsection{Cognitive resource demands of redirected walking}
\textbf{\cite{bruder2015cognitive}}\textsuperscript{**}

I am having issues to understand and feel the proper idea presented in this paper. But the basic that I understand is to analyze the effects of subtle continuous reorientation techniques based on curvature gains on cognitive load was the aim of the study. And, they showed that the radius of the circular path on which the users were redirected with curvature gains had a significant effect on verbal and spatial working memory.

\subsection[Subliminal reorientation and repositioning in immersive virtual environments using saccadic suppression]{Subliminal reorientation and repositioning\\ in immersive virtual environments using saccadic suppression}
\textbf{\cite{bolte2015subliminal}}

In this article, Bolte and Lappe demonstrated that subtle translation of the user's viewpoint can be applied during saccades (eye blinks). The aim of this study was to investigate whether saccadic suppression of image displacement (SSID) can be used in an immersive VE to unconsciously rotate and translate the observer's viewpoint. Their study concluded that users were unable to detect approximately $\pm 0.5$m translations along the line of gaze and $\pm 5^{\circ}$ rotations in the transverse plane during saccades with and amplitude of $15^{\circ}$.

\subsection[Towards physically interactive virtual environments: Reactive alignment with redirected walking]{Towards physically interactive\\ virtual environments: Reactive alignment with redirected walking}
\textbf{\cite{thomas2020towards}}\textsuperscript{**}

In this paper, Thomas et al. introduced a novel use of redirected walking (\emph{alignment}) that has the potential to increase usability and interactivity in VR applications. Recent research studies environmental alignment with much care to use RDW techniques to align virtual and physical coordinate systems so that it enables interactivity with the physical environment. With their reactive algorithm, Thomas et al. used repulsive mechanism to steer the user away from the physical obstacles. I think I need more time to grow my foundation and to understand the core of this paper properly.

\section{My Notes}

The following four criterion are required for the ideal redirected walking technique (though no single RDW technique can satisfy these criterion until today):

\begin{itemize}[noitemsep]
    \item[---] \emph{Imperceptible}: The user is unaware that redirection is taking place
    \item[---] \emph{Safe}: The walker is prevented from leaving the tracking space and colliding with physical obstacles and other users
    \item[---] \emph{Generalizable}: It is applicable within any VE and with any number of users
    \item[---] \emph{Devoid of unwanted side effects}: It does not introduce cybersickness or interfere with primary and secondary tasks
\end{itemize}

\subsection{Subtle Manipulation of Gains}

The following four types of gains are used to manipulate the mapping between the user's real and virtual movement:

\begin{itemize}[noitemsep]
    \item[---] \emph{Rotation gains (user in stationary state)}: Identifying detection thresholds for rotation gains during head turns and full body turns. Gains applied in the same direction as head rotation rather than against head rotation are less likely to be noticed by users (\cite{jerald2008sensitivity}).
    \item[---] \emph{Translation gains (user moving forward)}
    \item[---] \emph{Curvature gains (user moving forward)}
    \item[---] \emph{Bending gains (user moving on a curve)}
\end{itemize}

\subsection{Subtle Manipulation of Virtual Architecture}

Suma et al. introduced two forms of redirection through manipulation of virtual architecture: \emph{change blindness} redirection used to present two virtual rooms in the same tracking space (\cite{suma2011leveraging}), and \emph{impossible spaces} used to present two overlapping rooms (\cite{suma2012impossible}).

\emph{Change blindness} is the perceptual phenomenon of one's inability to detect changes in the environment. This form of architectural manipulation enabled users to navigate a large dynamic virtual office building of approximately 219 sq. meters without leaving a tracking space of $4.3m \times 4.3m$.

\emph{Impossible spaces} is the technique of compressing larger virtual interior environments into smaller physical spaces by means of self-overlapping architecture.

\subsection{Overt Redirection}

Overt technique guides the user back toward the center of the tracking space without stopping the virtual experience. When the user reaches the bounds of the tracking space, it is not uncommon to use overt techniques of redirection for the safety of the user and equipment.

Seven league boots was introduced by Interrante et al. (\cite{interrante2007seven}). The technique was based on the user enabling and disabling perceptible translation gains by pressing a button on a hand-held wand.

Williams et al. (\cite{williams2007exploring}) introduced three overt techniques. They are: \emph{freeze-backup}, \emph{freeze-turn}, \emph{2:1 turn}.

Peck et al. introduced an object (called \emph{distractor}) in the VE for the user to visually focus on while the VE rotates during the user's head turns (\cite{peck2011evaluation}). They suggested that improving visual realism and adding sound positively influenced participants' sense of presence in the VE.

\subsection{Side Effects of RDW}

\subsubsection{Simulator Sickness}

There are reports of experiencing some sickness symptoms after an exposure to VE such as dry mouth, nausea, dizziness, visual aftereffects (flashbacks), pallor, sweating, ataxia, and even vomiting (\cite{razzaque2005redirected}). These phenomenons are called as \emph{Simulator Sickness} or \emph{Cybersickness}.

\newpage

\printbibliography

\end{document}
